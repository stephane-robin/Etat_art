% ==============
% PARAMETRAGES
% à compiler en pdfLaTeX
% ==============

% GENERAL
%  type de document rapport, chapitre commence en page impaire ou paire indifféremment
\documentclass[twoside,a4paper,11pt,frenchb,openany]{report}  
%  type de document, chapitre commence en page impaire
%\documentclass[twoside,a4paper,12pt,frenchb,openright]{report} 
\title{Etat de l'art du projet de recherche}
\author{Stephane Robin}
\date{\today}

% IMPORTATION DE LIBRAIRIES
\usepackage{amssymb}  % symboles
\usepackage{amsmath}  % symboles mathématiques
\usepackage{amsfonts}  % polices de caractères
\usepackage{amscd}
\usepackage{amsthm}  % symboles mathématiques pour redéfinir les théorèmes
\usepackage[all,cmtip]{xy}
\usepackage{array}  % tableaux
\usepackage[frenchb]{babel}  % langue française
\usepackage{bm}  % caractères grecs
\usepackage{calc}
\usepackage{enumitem} % listes
\usepackage{eurosym}  % symbole euro
\usepackage{euscript}
\usepackage{fancybox}  % boîtes
\usepackage{float}  % images flottantes
\usepackage[T1]{fontenc}  % LaTeX modele
\usepackage[top=3.1cm,bottom=2.7cm,left=2.2cm,right=2.2cm,dvips]{geometry}  % marges
\usepackage{graphicx}  % insertion images
\usepackage[utf8]{inputenc}  % accents
\usepackage{mathrsfs}  % symboles mathématiques
\usepackage{mathtools}  % outils mathématiques
\usepackage{mdframed} % box autour des theoremes
\usepackage{pst-plot, pstricks}
\usepackage{pstricks-add}
\usepackage{rotating}
\usepackage{srcltx}
\usepackage{textcomp}   % caracteres complementaires
\usepackage{titlesec}  % sections
\usepackage{titletoc}  % table de contents
%\usepackage[nottoc,notlof,notlot]{tocbibind}  % bibliothèque
\usepackage{verbatim}  % caracteres non interpretes
\usepackage{wrapfig}  % pour inserer les figures dans du texte

% COULEURS
\definecolor{couleurTitre}{RGB}{64,128,128}  % doit être défini avant xcolor
\definecolor{couleurUrl}{RGB}{127,62,0}
\usepackage{xcolor}  % couleurs
%\definecolor{amber}{rgb}{1.0,0.49,0.0} % couleur non utilisee
%\definecolor{greyish}{rgb}{52.0,160.0,157.0} % couleur non utilisee
%\definecolor{theoremeCouleur}{rgb}{224.0,90.0,67.0} % couleur non utilisee
	
% LISTES
\renewcommand{\labelitemi}{$\bullet$}  % symboles de listes
\frenchbsetup{StandardLists=true}  % style de bullets

% IMAGES
\usepackage{caption}  % insertion d'images
\usepackage[font=footnotesize,labelfont=bf]{caption}  % légendes des images
\renewcommand{\thefigure}{\arabic{section}.\arabic{figure}}  % numérotation des images
\usepackage{subcaption}

% HYPERLINKS
\usepackage{hyperref}
\hypersetup{
colorlinks=true,
linkcolor=couleurUrl,
citecolor=couleurUrl,
urlcolor=couleurUrl
}

% LIGNES
\usepackage{parskip}
%\setlength\parskip{\baselineskip} % joint a la suppression de l'espace horizontal
%\setlength{\parindent}{0cm} % supprime l'espace horizontal en debut de ligne

% HEADINGS AND FOOTERS
\usepackage{fancyhdr}  % entete
\pagestyle{fancy}  % la page accepte les entetes et pieds de page
\renewcommand{\headrulewidth}{0pt}
\fancyhead[L,R,C]{}
%\fancyhead[LE]{} % pages paires header gauche
%\fancyhead[CE]{} % pages paires header centre
%\fancyhead[RE]{Le théorème de Pythagore} % pages paires header droit
%\fancyhead[LO]{Le théorème de Pythagore} % pages impaires header gauche
%\fancyhead[CO]{} % pages impaires header centre
%\fancyhead[RO]{} % pages impaires header droit
%\fancyfoot[c]{\textcolor{gray}{\thepage}}  % pied de page
%\fancyfoot[L,R,C]{} % forcing footer empty

% REDEFINITION DU STYLE DE THEOREM
\newmdtheoremenv[  % definitions
linewidth=5,
leftline=true,
rightline=false,
bottomline=false,
topline=false,
leftmargin=0,
rightmargin=0,
backgroundcolor=couleurTitre!20,
linecolor=orange!70,
innertopmargin=21pt,
skipabove=\topskip,
ntheorem=true]{definition}{Définition}

\newmdtheoremenv[  % theoremes
linewidth=5,
leftline=true,
rightline=false,
bottomline=false,
topline=false,
leftmargin=0,
rightmargin=0,
backgroundcolor=couleurTitre!20,
linecolor=orange!70,
innertopmargin=10pt,
skipabove=\topskip,
ntheorem=true]{theorem}{}

\newmdtheoremenv[  % proprietes
linewidth=5,
leftline=true,
rightline=false,
bottomline=false,
topline=false,
leftmargin=0,
rightmargin=0,
backgroundcolor=couleurTitre!20,
linecolor=orange!70,
innertopmargin=21pt,
skipabove=\topskip,
ntheorem=true]{property}{Propriété}

\newmdtheoremenv[  % propositions
linewidth=5,
leftline=true,
rightline=false,
bottomline=false,
topline=false,
leftmargin=0,
rightmargin=0,
backgroundcolor=couleurTitre!20,
linecolor=couleurTitre!70,
innertopmargin=21pt,
skipabove=\topskip,
ntheorem=true]{proposition}{Proposition}

\newmdtheoremenv[  % corollaires
linewidth=5,
leftline=true,
rightline=false,
bottomline=false,
topline=false,
leftmargin=0,
rightmargin=0,
backgroundcolor=couleurTitre!20,
linecolor=couleurTitre!70,
innertopmargin=21pt,
skipabove=\topskip,
ntheorem=true]{corollary}{Corollaire}

\newmdtheoremenv[  % lemmes
linewidth=5,
leftline=true,
rightline=false,
bottomline=false,
topline=false,
leftmargin=0,
rightmargin=0,
backgroundcolor=couleurTitre!20,
linecolor=couleurTitre!70,
innertopmargin=21pt,
skipabove=\topskip,
ntheorem=true]{lemma}{Lemme}

% NUMEROTATION DES CHAPITRES-SECTIONS
\renewcommand{\theequation}{\arabic{chapter}.\arabic{equation}}  % equations
%\renewcommand{\theequation}{\thesection\arabic{equation}}  % equations
%\numberwithin{equation}{section}  % equations

%\renewcommand{\thepart}{\Alph{part}}  % parties
\renewcommand{\thechapter}{\arabic{chapter}.}  % chapitres
\renewcommand{\thesection}{\arabic{section}.}  % sections
\renewcommand{\thesubsection}{\arabic{section}.\arabic{subsection}.}  %  sous-sections
\renewcommand{\thesubsubsection}{\arabic{section}.\arabic{subsection}.\arabic{subsubsection}.}  % sous-sous-sections

\renewcommand{\thedefinition}{\arabic{chapter}.\arabic{definition}}  % definitions
\renewcommand{\theorem}{}  % theoremes
\renewcommand{\theproperty}{\arabic{chapter}.\arabic{property}}  % property
\renewcommand{\theproposition}{\arabic{chapter}.\arabic{proposition}}  % propositions
\renewcommand{\thecorollary}{\arabic{chapter}.\arabic{corollary}}  % corollaires
\renewcommand{\thelemma}{\arabic{chapter}.\arabic{lemma}}  % lemmes
\setcounter{secnumdepth}{4}  % profondeur de numérotation

% FORMAT DES SECTIONS-TITRES
\titleformat{\section}{\normalfont\normalsize\bfseries}{\textcolor{couleurTitre}{\thesection}}{1em}
{\normalfont\large\bfseries\scshape\textcolor{couleurTitre}} % format de titre de section
\titleformat{\subsection}{\normalfont\normalsize\bfseries}{\textcolor{couleurTitre}{\thesubsection}}{1em}
{\normalfont\normalsize\bfseries\textcolor{couleurTitre}}  % format de titre de sous-section
\titleformat{\subsubsection}{\normalfont\normalsize\bfseries\itshape}{\textcolor{couleurTitre}{\thesubsubsection}}{1em}
{\normalfont\normalsize\bfseries\itshape\textcolor{couleurTitre}}  % format de titre de sous-sous-section

\makeatletter

%\renewcommand{\@chapapp}{}  % le mot `chapitre'' n'apparait plus en titre de chapitre
\renewcommand\l@section{\@dottedtocline{1}{0em}{5.0em}}  % espacement dans le titre d'une section
\renewcommand\l@subsection{\@dottedtocline{1}{2.5em}{1.5em}}  % espacement dans le titre d'une sous-section
\renewcommand\l@subsubsection{\@dottedtocline{1}{5em}{2.5em}}  % espacement dans le titre d'une sous-sous-section

\makeatother

% ==============
% DEBUT DU DOCUMENT
% ==============

\begin{document}

\maketitle

\section{Introduction}

Le présent document cherche à définir les fondements sur lesquels va reposer le stage et en particulier de connaître l'état actuel des connaissances concernant l'analyse bioinformatique du locus CRISPR-Cas chez \textit{Mycobacterium tuberculosis}.

L'objectif du stage est de comprendre les liens entre les sept lignées de \textit{tuberculosis} et les alternances de gènes et espaceurs du locus CRISPR-Cas, au moyen d'outils développés au sein de l'équipe AND de l'Université de Franche Comté.

\section{Lexique}

\textbf{MTBC} : Mycobacterium Tuberculosis Complex.

\textbf{Phylogénie} : étude des liens entre espèces apparentées, permettant de retracer les principales étapes de l'évolution des organismes depuis un ancêtre commun.

\textbf{Epidémiologie} : discipline scientifique qui étudie les problèmes de santé dans les populations humaines, leur fréquence, leur géographie ainsi que les facteurs influants.

\textbf{Annotation des gènes} : processus permettant d'identifier l'emplacement des gènes dans l'ADN, de déterminer leurs fonctions et leurs possibles interactions.

\textbf{Génome} : ensemble de l'information génétique d'un organisme. Par extension, le génome se réfère aussi au support physique de cette information génétique, la macromolécule d'ADN.

\textbf{Séquençage du génome} : consiste, par des méthodes chimiques ou de biologie moléculaire, à déterminer l'ordre des nucléotides de l'ADN.

\textbf{bp} : une paire de base est l'appariement de 2 bases nucléiques situées sur 2 brins complémentaires d'ADN, reliées par des ponts d'hydrogène.

\textbf{SNP Single Nucleotide Polymorphism} : il s'agit de la variation ou polymorphisme d'une seule paire de bases du génome entre organismes d'une même espèce.

\textbf{Haplogroupe} : groupe possédant les mêmes caractères génétiques et partageant un ancêtre commun suivant une mutation SNP.

\textbf{Homologie} : similitude de caractères observée chez deux espères différentes, provenant de l'héritage d'un ancêtre commun. 

\textbf{Homoplasie} : similitude de caractères chez différentes espèces, qui ne provient pas d'un ancêtre commun, mais peut par exemple provenir d'une adaptation à l'environnement.

\textbf{Mitochondrie} : centrale énergétique des cellules qui contribue à la production d'ATP.

\textbf{Polymère} : macromolécule répétant un même motif structural.

\textbf{Enzyme de restriction} : protéine capable de couper un fragment d'ADN au niveau d'une séquence de nucléotides caractéristique appelée site de restriction. Chaque enzyme de restriction reconnaît ainsi un site spécifique.

% OUT OF AFRICA MIGRATION ========================================================
\section{Evolution de la branche humaine de la tuberculose}

Le développement des maladies s'adapte à la densité de population concernée. En effet, auprès d'une foule dense, les infections se répandent plus largement et deviennent plus virulentes, alors qu'auprès d'une population moins importante, elles ont une croissance plus faible, laissant place parfois à des périodes où les infections restent latentes.

Une période charnière dans l'histoire de l'humanité est la transition démographique du Néolithique, qui a vu il y a 10 000 ans, suite à l'apparition de l'agriculture et de l'élevage, un accroissement de la population, favorisant la naissance de nombreuses maladies. Les maladies humaines plus anciennes se développaient auprès de populations moins denses et produisaient des phases chroniques de latence et de réactivation permettant aux populations infectées de survivre.

La tuberculose conjugue ces deux modèles de maladie. En effet, elle a montré à travers les âges des périodes de réactivation, elle dépend fortement de la densité de population et son mode de transmission aérosol s'est parfaitement adapté aux foules.

% The global diversity of human-adapted MTBC

L'analyse phylogénique de Comas et al.\cite{comas} se base exclusivement sur l'étude du génome complet de toutes les lignées connues de tuberculose en utilisant les SNPs pour construire les relations entre les différentes branches. Les résultats obtenus rejoignent de précédentes études effectuées à partir d'autres marqueurs, et confirment l'existence de sept principales lignées de tuberculose. On remarque en particulier que plusieurs branches d'origine animale se sont regroupées avec la lignée 6 d'Afrique de L'Ouest 2, et que les lignées modernes 2, 3 et 4 d'Europe ont des origines proches. Par ailleurs, seuls 1,1 \% des SNPs sont homoplastiques, ce qui suggère que la structure de la tuberculose favorise les clonages plutôt que les recombinaisons entre branches.  

% African origin and co-divergence of MTBC with modern humans

L'étude phylogénique de Comas et al.\cite{comas} corrobore les connaissances actuelles selon lesquelles la tuberculose est originaire d'Afrique. Par ailleurs, s'appuyant sur les origines africaines de l'espèce humaine, elle cherche également à déterminer l'ancienneté de l'association entre la tuberculose et son hôte humain. L'analyse des divergences des génomes de la tuberculose est comparée à celle d'une arborescence génétique déjà établie à partir de mitochondries de l'être humain. 

% Age of the association of MTBC and humans

Les similitude relevées montrent que la tuberculose a infecté les premiers hommes d'Afrique. Pour aller plus loin, l'étude de Comas et al. a tenu compte de 3 dates importantes dans l'évolution de l'être humain qui ont été reportées sur l'analyse phylogénique de la tuberculose des lignées 5 et 6 d'Afrique de l'Ouest :

- l'émergence de l'homo sapiens correspondant au MTBC-185,\\
- l'émergence du halogroupe mitochondriaque de la lignée 3 chez l'homme correspondant au MTBC-70,\\
- le début de la transition démographique du Néolithique correspondant au MTBC-10.

La branche MTBC-70 a révélé des corrélations avec l'histoire de l'humanité, telle qu'elle a été décrite par l'archéologie, en montrant l'apparition des sept différentes lignées de tuberculose :

- il y a 73 000 ans, apparition des lignées 5 et 6 correspondant à une 1ère migration humaine importante vers l'Afrique de l'Ouest,\\
- il y a 67 000 ans, apparition de la lignée 1 correspondant à une migration humaine importante autour de l'Océan Indient,\\
- il y a 64 000 ans, apparition de la lignée 7 concernant une population qui est restée en Afrique ou est revenue en Afrique après une 1ère migration,\\
- il y a 46 000 ans, apparition de la lignée 4 correspondant à une migration humaine importante vers l'Europe,\\
- il y a 42 000 ans, apparition des lignées 2 et 3 correspondant à une migration humaine importante vers l'Asie de l'Est et l'Asie Centrale. 

En revanche, la branche MTBC-185 suggère l'apparition de mutations à partir de lignées africaines il y a 174 000 ans, c'est à dire que la dispersion de la tuberculose précèderait celle de l'homo sapiens.

% Neolithic co-expansion of MTBC and humans

Dans tous les cas, la tuberculose aurait infecté l'espèce humaine et évolué conjointement avec elle depuis 70 000 ans, mais son apparition serait antérieure à la transition démographique du Néolithique.

La base de données de tuberculose étudiée de façon probabiliste par Comas et al. montre que le Néolithique a fortement contribué à l'expansion de la maladie il y a 10 000 ans grâce à l'augmentation de la densité de population, à la probabilité de co-infection avec d'autres maladies dépendantes de la densité de population, et non pas grâce à la possibilité pour la tuberculose de muter d'une variété animale vers une variété humaine. En effet, l'analyse phylogénique de la tuberculose montre que les branches humaines ont divergé des branches animales avant le Néolithique.

Le Néolithique n'était pas la seule période où l'augmentation de la population fut importante, toutefois la concentration de population qui s'en est suivie a permis l'apparition auprès de la tuberculose de caractères fortement dépendants de la densité de population qu'elle affecte. Le Néolithique a donc marqué un tournant dans l'histoire de la tuberculose qui a alors commencé à conjuguer les deux modèles de maladie, dépendant de la densité de population et chronique par périodes de réactivation.

% The evolutionary history of MTBC at a regional scale

 % Conclusion

Il faut donc considérer que la co-existence de la tuberculose avec l'espèce humaine depuis des milliers d'années a conduit la maladie à s'adapter aux changements du génome humain et inversement. Les prochaines études sur la tuberculose devraient donc se concentrer sur des génomes complets de la tuberculose et de l'être humain choisis en rapport à leurs associations.

Par ailleurs, une étude récente de Perry S. et al.\cite{perry1, perry2} suggère que l'infection d'une organisme par l'Helicobacter Pylori pourrait protéger de la tuberculose sous sa forme active. A contrario, nous ne savons pas si la tuberculose latente pourrait protéger contre les ulcers et les cancer de l'estomac causés par l'Helicobacter Pylori.



%  GLOBAL EXPANSION OF MTBC LINEAGE 4 =============================================

\section{L'expansion de la lignée 4 de tuberculosis}





% SPOLPRED ==========================================================

\section{Le locus CRISPR-Cas}

% Le locus CRISPR Cas

\subsection{Description}

Le locus CRISPR \textit{Clusterd Regularly Interspaced Short Palindromic Repeats} est une famille de séquences répétées dans l'ADN formant un palindrome. Chaque série de répétition contient entre 21 et 37 bp, régulièrement espacées par des \textit{spacers} de 20 à 40 bp. Cette structure répétée se trouve à l'état naturel dans les bactéries et les archées. Elle est héritable par transmission aux cellules filles et se conserve donc pour une même espèce. Les locus CRISPR sont généralement adjacents aux gènes Cas, dont ils sont séparés par une séquence de 300 à 500 bp. Ces gènes Cas produisent des enzymes capable de couper l'ADN en vue de leur réparation.

Ces séquences incorporent dans les spacers des fragments d'ADN de bactériophages qui ont déjà infecté la bactérie qui sont stockés en vue de détecter et de détruire l'ADN de bactériophages similaires en cas d'infection ultérieure. Par conséquent, CRISPR-Cas est un système immunitaire naturel utilisé par les bactéries pour se protéger des infections virales. 

% dessin




\subsection{Fonctionnement du système CRISPR-Cas}

Le système CRISPR-Cas fonctionne de la façon suivante : lorsque la bactérie détecte la présence d'ADN ou d'ARN d'un virus, elle produit une protéine appelée Cas9 capable de couper l'ADN, une séquence d'ARN CRISPR notée crARN correspondant à celle de l'ADN du virus et une séquence d'ARN traceur notée trARN qui joue le rôle d'un guide ARN. Lorsque trARN trouve sa cible parmi le genome du virus, Cas9 sectionne son ADN, désactive le virus, puis insère un fragment de l'ADN du virus dans un \textit{spacer} du génome de la bactérie afin de conserver en mémoire une trace de ce virus.

Les différentes étapes du système CRISPR-Cas sont donc les suivantes :

1. adaptation de l'ADN de la bactérie à l'ADN ou ARN du bactériophage, en incorporant des portions d'ADN du virus dans les spacers de la bactérie,\\
2. création de crRNA et trRNA à partir de la nouvelle séquence d'ADN légèrement modifiée suite à l'infection virale. Avec l'aide de Cas9, trRNA et crRNA coupent l'ADN cible à un emplacement spécifique.\\
3. création d'enzymes permettant de lutter contre le bactériophage.

Mécanisme moléculaire :

Les systèmes CRISPR utilisent les gènes cas1 et cas2 qui sont impliqués dans l'intégration, en tant que spacer de fragments de gènes étrangers dans le CRISPR.

Trois types de systèmes CRISPR-Cas sont connus :

les systèmes de types I, utilisent un complexe Cascade pour cliver les transcrits de CRISPR au niveau des épingles. Lorsqu'un complexe Cascade/spacer s'associe à un ADN cible (reconnaissance par hybridations) il recrute la protéine Cas3 qui clive un brin de l'ADN cible ;
les systèmes de types II, utilisent la RNAse III pour séparer les répétitions des transcrits. La protéine Cas9 s'associent avec un fragment de transcrit et, lors de la reconnaissance d'un ADN cible, Cas9 clive les 2 brins de cet ADN ;
les systèmes de type III, utilisent la protéine Cas6 pour cliver les transcrits de CRISPR au niveau des épingles, les segments de transcrits obtenus s'associent avec un complexe Cas10. Ce système requiert qu'il y ait transcription de l'ADN cible, le complexe Cas10/spacer clive alors un brin de l'ADN cible (brin non transcrit), ainsi que l'ARN en cours de transcription.

% REVOIR VIDEO POUR COMPARER

La technologie CRISPR-Cas9 s'inspire du système du même nom a d'abord été utilisée pour typer les souches bactériennes, suivant une technique appelée \textit{spoligotyping}. CRISPR-Cas9 est actuellement principalement employé comme ciseau moléculaire afin d'éditer le génome et d'introduire localement des modifications génétiques.

% Spoligotyping

\subsection{Le spoligotyping}

Une PCR \textit{Polymerase Chain Reaction} est une méthode de réaction en chaîne utilisant un polymère pour dupliquer en grand nombre une séquence d'ADN spécifique. La méthode PCR repose sur le cycle thermique, qui expose les séquences à des cycles répétés de chauffage et de refroidissement pour permettre différentes réactions dépendantes de la température comme la fusion de l'ADN et la réplication de l'ADN par les enzymes. La méthode PCR utilise deux agents principaux : les polymères d'ADN et les \textit{primers}.

Le spoligotyping, \textit{Spacer Oligonucleotide Typing}, repose sur la détection de séquences répétitives, ou \textit{Direct Repeat} DR, trouvées entre les gènes d'un agent infectieux au sein d'un locus CRISPR, et permet le génotypage des souches de tuberculosis en utilisant la duplication des séquences d'ADN à l'aide d'une méthode PCR.

Les données obtenues sous forme numérique peuvent être partagées entre laboratoires et corroborent les résultats obtenus à partir d'autres marqueurs génétiques. Elles permettent de bien différencier les souches de tuberculosis et sont de moindre coût comparativement à d'autres méthodes. Cependant, le spoligotyping éprouve des difficultés à bien différencier les souches au sein de grandes familles de souches telles que la branche Beijing.

Le spoligotyping a permis de fournir une image globale de la diversité des souches de tuberculosis.

% Spolpred

\subsection{Quel outil choisir pour le spoligotyping ?}

SpolPred est un logiciel de prédiction rapide et précis des spoligotypes de tuberculosis à partir de séquences génomiques courtes appelées reads, développé par Preston M.

Dans son étude, Coll F. et al.\cite{coll} montre l'utilité de ce logiciel mais aussi les limites de la méthode bioinformatique par assemblage, consistant à aligner ou fusionner des fragments d'ADN issus d'une plus longue séquence afin de reconstruire la séquence originale. Il apparaît que SpolPred offre plus de précision, plus de rapidité et des résultats pratiquement identiques à ceux obtenus par assemblage.

Une problématique de SpolPred en 2020 est que le logiciel n'est plus disponible au public. En effet, une visite sur le site officiel \url{http://www.pathogenseq.org/spolpred} fourni comme référence dans le document \cite{coll} de Coll F. et al. montre que le nom de domaine est à vendre. Preston M., qui a fait partie de l'équipe de rechercher de Coll F. pour le développement de SpolPred, a bien créé un site \url{https://www.mybiosoftware.com/spolpred-predict-the-spoligotype-from-raw-sequence-reads.html} proposant le téléchargement du logiciel, mais le lien est actuellement inactif.

Une alternative à SpolPred est SpoTyping présenté dans l'article \cite{xia} de Xia et al. comme étant 20 à 40 fois plus rapide que SpolPred pour prédire avec précision des spoligotypes de tuberculosis à partir de reads de taille uniforme ou variable. Par ailleurs, SpoTyping produit un rapport résumant les données épidémiologiques associées à l'étude à partir d'une base de données mondiale de tous les isolats ayant le même spoligotype. SpoTyping peut se télécharger gratuitement à l'adresse \url{https://github.com/xiaeryu/SpoTyping-v2.0}.
% dependancies
SpoTyping utilise

La base de données SITVITWEB contient 2740 types partagées, appelés SITs, parmis 58180 isolats et 62 branches. 

L'application en ligne SitVit2 permet d'analyser des data au regard des markers Spoligotypes, ETR et MIRU-VNTR afin d'en trouver notamment l'origine.


% finir lecture spotyping






Une méthode standardisée de description des spoligotypes a été proposée par Dale JW dans son article \cite{dale}. L'adoption de cette méthode pourrait s'avérer utile à la recherche mycobactérienne.

Dale JW et al. explique que la séquence répétée DR au sein du locus CRISPR de tuberculosis mesure 36 bp. Les spacers, quant à eux mesurent de 35 à 41 bp. Le polymorphisme entre les différentes souches résulte des variations et de l'identité des spacers. La région DR d'un isolat devant être testé est agrandi par PCR pour dévoiler un motif de taches correspondant aux spacers. La comparaison de ces motifs permet la différentiation des souches. A l'heure actuelle, il n'existe pas de norme pour décrire ces motifs ou simplement les numéroter. Chaque laboratoire utilise son propre système de numérotation accompagné d'un schéma descriptif du motif. Ce manque de normalisation entrave les possibilités de comparaison des résultats obtenus et le développement d'une vision mondiale de l'évolution de MTBC.

\subsection{Vers une normalisation des spoligotypes}

Une base de données centralisée regroupant les 1362 motifs connus et de leurs numérotation jusqu'à présent associées existe au RIVM Rijksinstituut voor Volksgezondheid en Milieuhygiene, Bilthoven, Netherlands, et peut être consultée au \url{http://www.caontb.rivm.nl}. Les nouveaux motifs devraient alors prendre un unique numéro pour être répertoriés dans cette base de données.

Toutefois, cela nécessite l'interrogation systématique de la base de données et la comparaison avec les éléments déjà existant pour chaque nouveau spoligotype. Pour éviter cette perte de temps, de nombreux laboratoires utilisent des systèmes rationnels avec des codes descriptifs assignés à chaque isolat.
%finir lecture standard














% TODO =========================================

% chercher à quoi ressemblent des illumina reads, éventuellement comprendre le fonctionnement de illumina sequencing



% =============================================



% =========
% BIBLIOTHEQUE
% =========
\bibliographystyle{}
\renewcommand{\bibname}{Références}
\begin{thebibliography}{10}
\bibitem{comas}Comas I. et al. \textit{Out-of-Africa migration and Neolithic coexpansion of Mycobacterium tuberculosis with modern humans}\\ \\

\bibitem{coll}Coll F. et al., \textit{SpolPred : rapid and accurate prediction of Mycobacterium tuberculosis spoligotypes from short genomic sequences}\\ \\

\bibitem{brynildsrud}Brynildsrud O.B. et al., \textit{Global expansion of Mycobacterium tuberculosis lineage 4 shaped by colonial migration and local adaptation}\\ \\

\bibitem{driscoll}Driscoll J. R., \textit{Spoligotyping for molecular epidemiology of the Mycobacterium tuberculosis complex}\\ \\

\bibitem{jinek}Jinek M. et al, \textit{A programmable dual-RNA-guided DNA endonuclease in adaptive bacterial immunity}\\ \\

\bibitem{gori}Gori A. et al, \textit{Spoligotyping and Mycobacterium tuberculosis}\\ \\

\bibitem{perry1}Perry S. et al., \textit{Infection with Helicobacter pylori is associated with protection against tuberculosis}\\ \\

\bibitem{perry2}Perry S. et al, \textit{The immune response to tuberculosis infection in the setting of Helicobacter pylori and helminth infections}\\ \\

\bibitem{xia}Xia E. et al., \textit{SpoTyping: fast and accurate in silico Mycobacterium spoligotyping from sequence reads}\\ \\ 

\bibitem{dale}Dale JW. et al., \textit{Spacer oligonucleotide typing of bacteria of the Mycobacterium tuberculosis complex: recommendations for standardised nomenclature}\\ \\



\end{thebibliography}
\end{document}